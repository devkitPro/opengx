\documentclass[12pt]{article}

\usepackage{subfigure}
\usepackage{graphics}
\usepackage{graphicx}
\usepackage{color}
\usepackage[rgb]{xcolor}
\usepackage{hyperref}
\usepackage{listings}
\lstset{language=C}

\title{\textbf{\huge{OpenGX}}}
\author{
    David Guillen Fandos \\ \small{(david@davidgf.net)}
    \and
    Alberto Mardegan \\ \small{(info@mardy.it)}
}
\date{\today}
\usepackage{amsmath}
\usepackage{float}

\lstset{
	tabsize=4,
	language=matlab,
        basicstyle=\scriptsize,
        aboveskip={1.5\baselineskip},
        columns=fixed,
        showstringspaces=false,
        extendedchars=true,
        breaklines=true,
        prebreak = \raisebox{0ex}[0ex][0ex]{\ensuremath{\hookleftarrow}},
	frame=single,
        showtabs=false,
        showspaces=false,
        showstringspaces=false,
        identifierstyle=\ttfamily,
        keywordstyle=\color[rgb]{0,0,1},
        commentstyle=\color[rgb]{0.133,0.545,0.133},
        stringstyle=\color[rgb]{0.627,0.126,0.941},
	language=C
}

\newcommand{\fname}[1] {{\color{blue}#1}}

\begin{document}

\maketitle

\section{Introduction}
On summer 2009 I started working on a PSP game project as a result of the console being hacked on the previous months. The idea of Toy Wars was conceived and the working started. My idea was to build a multiplatform game because that gave some advantages over console specific games. I chose OpenGL and SDL as the main libraries because they were available on the platform thanks to the scene contributions. This way I could create a computer game which was compatible with PSP.

After some time it became a decent game (for a homebrew) and I got tired of the PSP scene. I wanted to expand my horizons on console development and searched for a new target to focus on. I had been working on other consoles previously (in 2007-2008), Dreamcast and GB/GBA. Those projects didn't lead to any results (apart from two demos) but gave me some great experiences because their corresponding scenes were long dead, thus requiring extra effort to get into them.

So I found my brother's old Game Cube lying in the wardrobe's bottom. Immediately I had the need to start working on the scene although it was an old console and probably the scene would be dead. But surprisingly for me I found that the scene was quite alive. The secret was (I found out later) that the GC's successor, the Wii, was a cheap copy with some simple additions, so the Wii scene was a natural successor of GC's.

I started the port, which I thought it would be pretty straightforward, but after some time I realized that it would need more time to successfully port Toy Wars. The main problem was the lack of "standard" libraries such as SDL and OpenGL. The GC and Wii scene was and old one and nobody cared about using multiplatform libraries which I guess it's common on console development. The SDL port for Wii existed and was quite good, but no port existed for the GC. Hopefully I found that there was a guy working on a GC port. I mailed him and he was kindly enough to send me his sources which, I have to say, worked like a charm.

OpenGX was the remaining piece of the puzzle. I needed an OpenGL renderer and the only project which existed by the time was a complete non-working mess. So instead of changing the render code I wrote an OpenGL wrapper so it could be used with other projects. That is the way OpenGX was born.

\section{Aim and scope}

In this document I want to explain how GX works and how OpenGX uses it to emulate an OpenGL pipeline. GX is messy and a little difficult to understand, so I expect that this contribution helps to lighten some of the darkest areas.

It isn't the aim of the document to be exhaustive about the GX pipeline so, for a fully understanding of it, refer to the official GX documentation \cite{nintendogx} and the free libOGC GX implementation documentation \cite{libogc_gx_docs}

Also the document won't cover all the OpenGX functionality but it's core and most interesting aspects (the ones which I've found more instructive to talk about). For the complete reference it's a good idea to look at the source code.

\section{Introduction to GX's pipeline}

The GX pipeline can be divided in two big parts: the geometry processing and the pixel processing. The first one takes care of the operations in the "vertex domain", which involves transformation of the vertices, projection and clipping. The pixel processing is responsible for pixel rasterization, which includes color, texture and lighting calculations. To be exact lighting is partly calculated by the geometry unit.

Figure~\ref{fig:gx_full_pipeline} shows a sketch of the pipeline stages.

\begin{figure}[ht]
\centering
\includegraphics[width=0.7 \textwidth]{pipeline.png}
\caption{Sketch of the GX pipeline. Left stages care about geometry processing while the right ones carry pixel calculations}
\label{fig:gx_full_pipeline}
\end{figure}


\subsection{Vertex calculations}

The vertex calculation follows a classic scheme. We have vertices, normals and texture coordinates which can be transformed using some matrices.

The model-view matrix transforms the vertices in model-world space to view space. The projection matrix then transforms the vertices in view space to projection space (2D space, the screen). We also have a normal matrix, for transforming the normals and a texture transform matrix for transforming texture coordinates within its space.

The main differences in the process is how the matrices are represented. First of all OpenGL uses column-major matrices while GX uses row-major matrices. This is OpenGL stores columns contiguously in the memory while GX stores rows contiguously in the memory. Also the matrices have different size. The projection matrix is 4x4 but the modelview matrix is a 3x4 matrix (as the last row is never used for three component vector transformations). The normal matrix is a 3x4 matrix which only uses the 3x3 submatrix (ignoring the last column) because normals can only be rotated, not translated. The texture transformation matrix can be either a 2x4 matrix or a 3x4 matrix, depending on the usage.


\begin{figure}[ht]
\centering
\subfigure[Projection matrix] {
$ \begin{pmatrix}
	11 & 12 & 13 & 14\\
	21 & 22 & 23 & 24\\
	31 & 32 & 33 & 34\\
	41 & 42 & 43 & 44\\
\end{pmatrix} $
}
\hspace{1em}
\subfigure [Model-view matrix]{
\centering
$ \begin{pmatrix}
	11 & 12 & 13 & 14\\
	21 & 22 & 23 & 24\\
	31 & 32 & 33 & 34\\
\end{pmatrix} $
}
\hspace{1em}
\subfigure [Normal matrix]{
\centering
$ \begin{pmatrix}
	11 & 12 & 13 & {\color{gray}14}\\
	21 & 22 & 23 & {\color{gray}24}\\
	31 & 32 & 33 & {\color{gray}34}\\
\end{pmatrix} $
}
\\
\subfigure[Texture coordinates matrices] {
\centering
$ \begin{pmatrix}
	11 & 12 & 13 & {\color{gray}14}\\
	21 & 22 & 23 & {\color{gray}24}\\
\end{pmatrix} $
\hspace{2em}
$ \begin{pmatrix}
	11 & 12 & {\color{gray}13} & {\color{gray}14}\\
	21 & 22 & {\color{gray}23} & {\color{gray}24}\\
\end{pmatrix} $
}
\caption{Matrices involved in vertex transformations. Greyed out columns represent elements of the matrix which are not used by the calculations but must be present in memory }
\label{fig:matrices}
\end{figure}

The hardware is capable of storing up to ten model-view and normal matrices, one projection matrix and ten texture matrices. The matrices must be loaded and flushed to the GPU pipeline before they can be used, so it's a good idea to have some slots to load matrices which never change or are used frequently. The hardware allows to switch between matrix slots very easily (and fast).

\subsection{Pixel calculations}

After the vertex transform, projection and clipping it all comes to pixel work. The hardware calculates the resulting color of every pixel for each piece of geometry sent to the GPU. The pixel is tested against the depth component to determine its visibility and then may be rendered on the screen.

The process of determining the pixel color is very complex because the GX allows the programmer to control the fixed pipeline in a similar way pixel or fragment shaders do. The final pixel color is calculated using a custom combination of the rasterized color (from vertex and lighting), the texture color and constant colors of our own choose. This combination occurs right before fogging and blending.

The operation is calculated in a \emph{Texture Environment} unit (TEV), which is a combinational circuit which has colors as entries and a color output. The unit calculates the output color using up to four input colors as can be seen in Figure~\ref{fig:gx_tev_op}. The programmer can choose which operation (\emph{OP}) wants to perform, which \emph{scale} factor wants to apply to the color and an optional additive \emph{bias}. The functions \fname{GX\_SetTevColorOp} and \fname{GX\_SetTevAlphaOp} are used to setup all the named parameters.

\begin{figure}[ht]
\centering
\includegraphics[width=0.7 \textwidth]{stages.png}
\caption{Diagram of the Texture environment unit, which calculates a color using up to four input colors }
\label{fig:gx_tev_op}
\end{figure}

The TEV unit can be reused multiple times for the same screen pixel, so the operation that can be computed can be very complex. To achieve this unit re-usage the output is connected to four registers and the input is connected to those for registers as well as other inputs. This allows the programmer to use the output of previous calculations as the input of another calculation. The function \fname{GX\_SetNumTevStages} sets the number of stages.

The unit can be used up to sixteen times for the same pixel. The less cycles needed for a pixel calculation the fastest the rendering will be. An example is shown in Figure~\ref{fig:tev_stage_sample1}


\begin{figure}[ht]
\centering
\subfigure [Example of a color operation using three TEVs]{
\includegraphics[width=0.6 \textwidth]{tev_sample1.png}
}
\subfigure [Implementation of the operation using one TEV thrice (using auxiliary registers and thus taking three cycles)]{
\includegraphics[width=0.6 \textwidth]{tev_sample2.png}
}
\caption{Example of a color operation which is calculated using a TEV unit and using the four auxiliary registers for intermediate data. We are using three \emph{TEV stages}}
\label{fig:tev_stage_sample1}
\end {figure}

So, what inputs can be used in a TEV stage (not counting the four auxiliary registers)? We can use the color from a texture (in fact multiple textures), the color from the rasterizer (which is calculated using lighting equations) and some constant registers.

\begin{figure}[ht]
\centering
\subfigure [Inputs available for the four TEV inputs]{
\includegraphics[width=0.5 \textwidth]{inputs.png}
}
\hspace{1em}
\subfigure [Possible output registers for a TEV]{
\includegraphics[width=0.2 \textwidth]{outputs.png}
}
\caption{Diagram showing the possible inputs and outputs for a TEV stage. It's possible to choose between color or alpha channel in some inputs}
\label{fig:tev_stage_sample2}
\end {figure}

The four registers are named \emph{reg0}, \emph{reg1}, \emph{reg2} and \emph{prev}. When concatenating TEV stages it's common to use the \emph{prev} register. Also the \emph{prev} register is used to store the final pixel color and alpha, so the last TEV stage should have \emph{prev} register as its output. The functions \fname{GX\_SetTevColorIn} and \fname{GX\_SetTevAlphaIn} set the inputs for each stage.

The TEX input represents the rasterized texture color. The rasterization is done using the texture coordinates and the color is calculated in the texture unit (which applies decompression and/or filtering if required) and passed to the stage as an input. There's only one texture input so in theory only one texture can be rasterized. But the hardware allows to rasterize multiple textures at each TEV stage. So if we want to combine two textures in the TEV we \emph{must} use two stages. The first stage would retrieve the first texture color and the second stage would retrieve the other and combine them.

The RAS input is the color outputted from the rasterizer. This color is calculated using lighting equations, so it takes in account the materials, lights and vertex colors. We'll discuss extensively as it's one of the most important inputs.

The KONST input provides a constant color which can be specified by the programmer for each stage. There are hardwired colors but also four registers for the programmer to write. Those registers can be written using \fname{GX\_SetTevKColor}. For each stage the constant color to use can be selected with the function \fname{GX\_SetTevKColorSel}.

In addition to the constant colors provided by the input KONT, the inputs ZERO, ONE and HALF can be selected as inputs for a TEV stage. This inputs, as their names suggest, are hardwired values to zero, one and 0.5f. They're useful (specially zero) for performing simple operations which only require two or three operands.

\subsubsection{ Texture rasterization }

Each TEV stage can only lookup one texture pixel, as can be seen in Figure~\ref{fig:tev_stage_sample2}. So we can control which texture is going to be rasterized in each TEV stage. The function \fname{GX\_SetTevOrder} allows to specify which texture is going to be rasterized and presented at the TEX input. Also we must specify the texture coordinates that will be used for the rasterization (as vertices can have up to eight texture coordinates).

As an example we are going to consider the rasterization of a lightmapped scene. In a lightmapped scene we have two textures (the colored one and the lightmap) and two pairs of texture coordinates. The operation we are going to perform is to modulate the colors, that is, multiplying the colors components so that the original color gets darkened or lightened depending on the gray amount. A DirectX 6 example can be found at \cite{dx6_mt} with the explanation.

First of all we have to setup two TEV stages and indicate that two sets of texture coordinates are being used.

\begin{lstlisting}[frame=single]
GX_SetNumTevStages(2);
GX_SetNumTexGens(2);
\end{lstlisting}

Now we're going to setup the first TEV stage, which will select the TEX as input D and zero for the other inputs. The operation doesn't matter at all.

\begin{lstlisting}[frame=single]
GX_SetTevColorIn(GX_TEVSTAGE0,GX_CC_ZERO,GX_CC_ZERO,GX_CC_ZERO,
					GX_CC_TEXC);
GX_SetTevAlphaIn(GX_TEVSTAGE0,GX_CA_ZERO,GX_CA_ZERO,GX_CA_ZERO,
					GX_CA_TEXA);

GX_SetTevColorOp (GX_TEVSTAGE0, GX_TEV_ADD, GX_TB_ZERO, GX_CS_SCALE_1,
					GX_TRUE,GX_TEVPREV);
GX_SetTevAlphaOp (GX_TEVSTAGE0, GX_TEV_ADD, GX_TB_ZERO, GX_CS_SCALE_1,
					GX_TRUE,GX_TEVPREV);
\end{lstlisting}


We are storing the result in the \emph{prev} register. For the next stage we will use inputs B and C and ignore the other (zeroed). B will be the previous value (texture 1 color) and C will be texture 2 color.

\begin{lstlisting}[frame=single]
GX_SetTevColorIn(GX_TEVSTAGE1, GX_CC_ZERO, GX_CC_CPREV, GX_CC_TEXC,
					GX_CC_ZERO);
GX_SetTevAlphaIn(GX_TEVSTAGE0, GX_CA_ZERO, GX_CA_APREV, GX_CA_TEXA,
					GX_CA_ZERO);

GX_SetTevColorOp (GX_TEVSTAGE1, GX_TEV_ADD, GX_TB_ZERO, GX_CS_SCALE_1,
					GX_TRUE, GX_TEVPREV);
GX_SetTevAlphaOp (GX_TEVSTAGE1, GX_TEV_ADD, GX_TB_ZERO, GX_CS_SCALE_1,
					GX_TRUE, GX_TEVPREV);
\end{lstlisting}

Now we have to select the texture coordinates used in each stage. We have to select one texture coordinate slot (of the eight available)  and the transform matrix. It's possible to use a hardwired identity matrix, so there's no need to upload one.

\begin{lstlisting}[frame=single]
GX_SetTexCoordGen(GX_TEXCOORD0, GX_TG_MTX2x4, GX_TG_TEX0, GX_IDENTITY);
GX_SetTexCoordGen(GX_TEXCOORD1, GX_TG_MTX2x4, GX_TG_TEX1, GX_IDENTITY);
\end{lstlisting}

Now we have to setup the TEX input so the rasterized colors from the previous slots are passed to the TEVs. The operation selects the texture coordinates slots and the texture itself (the texture load code is not shown as it's very large and doesn't affect this explanation).

\begin{lstlisting}[frame=single]
GX_SetTevOrder(GX_TEVSTAGE0, GX_TEXCOORD0, GX_TEXMAP0, GX_COLORNULL);
GX_SetTevOrder(GX_TEVSTAGE1, GX_TEXCOORD1, GX_TEXMAP1, GX_COLORNULL);
\end{lstlisting}

The diagram representing the previous steps is shown in Figure~\ref{fig:lm_sample}


\begin{figure}[ht]
\centering
\includegraphics[width=0.70 \textwidth]{lm_sample.png}
\caption{Example showing a simple lightmap renderer using two textures and no lighting. The first stage just passes the TEX input to the output while the texture colors are multiplied in the second stage}
\label{fig:lm_sample}
\end {figure}

\pagebreak
\subsubsection{ Color rasterization }

The rasterized color is presented to a TEV stage through the RAS input (Figure~\ref{fig:tev_stage_sample2}). It's possible two choose between to color channels named COLOR0A0 and COLOR1A1. To specify which channel to present to the TEV stage we use \fname{GX\_SetTevOrder}.

When lighting is disabled the rasterized color can be either the content of a register or the rasterized vertex color. The register is called Material Color and can be specified with the function \fname{GX\_SetChanMatColor}. In order to choose between them we use \fname{GX\_SetChanCtrl} passing GX\_SRC\_REG or GX\_SRC\_VTX. Figure~\ref{fig:nolight} shows a circuit describing the color calculation process.

In order to be able to use two color channels it's mandatory to supply two vertex colors. If only one color is supplied the color is passed to channel number zero, even if the color is supplied as channel 1 color.

\begin{figure}[ht]
\centering
\includegraphics[width=0.68 \textwidth]{nolight.png}
\caption{Calculation of the RAS color present at the input of a TEV stage when no lighting is used. It's necessary to choose the color channel from the two available and the color source, which can be register or vertex rasterized}
\label{fig:nolight}
\end {figure}

Lighting adds complexity to the calculations. In fact the unlit model is a particular case of the lit model with brightness at its maximum value. The final color its calculated as follows (everything are colors and the dot product is a component product):

$$ Color = Material \cdot LightFunc $$

Where $Material$ is the unlit color (the one which comes from Material register or rasterized vertex color) and LightFunc the amount of lighting. With lighting disabled $LightFunc$ is white color, so that $Color$ equals $Material$.

The $LightFunc$ is calculated as follows:

$$LightFunc = Ambient + $$
$$ \sum_{i=0}^7 LightEnabled_i \cdot Attenuation_i \cdot DiffuseAttenuation_i \cdot Color_i $$

The $Ambient$ term is a color which can come from the vertex rasterized color or a register (like $Material$). The register value can be specified using \fname{GX\_SetChanAmbColor}. Each light can be enabled or disabled for a given channel, so if the light is disabled it doesn't contribute to the lighting term.

The attenuation term is evaluated from zero to one indication how the light contributes to the vertex illumination. It depends with the vertex-light distance and can be controlled with the function family \fname{GX\_InitLightAttn*}, which allow to specify the attenuation function in a very flexible way (directional, positional, spotlight, etc.). It's important to specify the light position and direction using \fname{GX\_InitLightDir} and \fname{GX\_InitLightPos}.

The diffuse attenuation is a term calculated taking into account the angle between the light and the surface normal, so it lights the polygon depending on its angle. It's controlled with \fname{GX\_SetChanCtrl}, so it's channel specific. This means that it's not possible to mix spot lights and positional or directional lights in the same channel.

The $Color$ term is the color of the light and can be different for every light. It can be specified with \fname{GX\_InitLightColor}.

Figure~\ref{fig:lights} represents the lighting calculations.

\begin{figure}[ht]
\centering
\includegraphics[width=0.95 \textwidth]{light.png}
\caption{Schematic representing lighting calculation process. }
\label{fig:lights}
\end {figure}

When lighting is disabled the lower subsystem is disabled so that the output color is calculated from the material color. Note also that the diffuse term depends on the channel number instead of the lighting object.


\section{OpenGX implementation}

GX and OpenGL are both finite state machines. But, since there are many diferences between them, I haven't established a relationship between their states. Instead I've decided to keep the GL state in memory and update GX state accordingly when needed.

The OpenGX state is stored in memory and includes: transform matrices, state bits, textures, geometry data (pointers or data itself), lighting data and any other status data (current color, texture, etc.). This state is used to update the GPU state when needed. In order to avoid unnecessary transfers the state also includes some dirty bits, which indicate the need to update certain information. When GL state changes some bits may become dirty and the rendering process cleans those bits by transferring and updating the state.

The immediate mode is implemented using data pointers. There are pointers for each data set: vertices, indices, normals, colors and texture coordinates. The \fname{glVertex*} calls are implemented by copying the data to a temporary buffer and using \fname{glDrawArrays} function to draw the data (at \fname{glEnd}). Each pointer has an integer value indicating the stride for the data in the array. The \fname{glInterleavedArrays} function is emulated using \fname{gl*Pointer} and \fname{glDrawArrays} functions with the adequate pointers and strides.

The rendering process consists in a setup stage and a loop for the data transfer. This data transfer consists in writing vertex data into GX's FIFO. The data is written interleaving its components (when no indices are used), so using interleaved buffers should be faster than separated buffers rendering (because of the cache). The data present in the arrays is used selectively depending on the context: if texturing is disabled the texture coordinates are ignored and not sent to the GPU (which is faster). Also if lighting is disabled normals are not sent too. A little optimization is performed for vertex colors: if the vertex color remains constant no color data is sent through the FIFO and the GPU is instructed to use the constant color register.

\subsection{Rendering cases}

The implemented render setup depends in two main factors: lighting and texturing.

\subsubsection{Unlit and untextured render}

When the lights and textures are disabled the color of the pixels is provided by the vertex colors. In this case the setup will use one TEV stage which takes the RAS input to the output (PREV register).

\begin{figure}[ht]
\centering
\includegraphics[width=0.40 \textwidth]{case1.png}
\caption{Unlit and untextured case. }
\label{fig:case1}
\end {figure}

\subsubsection{Unlit and textured render}

The final pixel color in this case is the product between the rasterized color and the texture color. So only one TEV stage is used to multiply the RAS input with the TEX input.

\begin{figure}[ht]
\centering
\includegraphics[width=0.50 \textwidth]{case2.png}
\caption{Unlit and textured case. }
\label{fig:case2}
\end {figure}

\subsubsection{Lit and untextured render}

The light color is calculated using two channel colors. We're going to use channel 0 to emulate ambient and specular lights and channel 1 to emulate diffuse lights. This decision is explained later. So we'll use two TEV stages, the first for rasterizing the channel zero and the second for rasterizing the channel one and combine them.

\begin{figure}[ht]
\centering
\includegraphics[width=0.55 \textwidth]{case3.png}
\caption{Lit and untextured case. }
\label{fig:case3}
\end {figure}

\subsubsection{Lit and textured render}

Adding texturing to the lit case is just like adding texturing to the unlit case: just multiply the result color by the texture color. This is done by adding an extra stage at the end of the pipeline to multiply the color.

\begin{figure}[ht]
\centering
\includegraphics[width=0.8 \textwidth]{case4.png}
\caption{Unlit and untextured case. }
\label{fig:case4}
\end {figure}

As the time of writting I've just realized that it's possible to use only two stages by multiplying the texture color to each color channel before adding them. Anyway this would be an optimization.

\subsection {Lighting management}

OpenGL has support for up to eight lights. Each light has a direction, position, cutoff, etc. depending how the light behaves (it's a point, a light cone, a directional light...). But all the lights have three color components: ambient, diffuse and specular. The ambient component only depends on the distance of the light and the vertex. The diffuse component depends on the distance (just like the ambient) but also depends on the light angle formed by the light and the surface normal. Finally specular lights depend on the distance as well and the angle formed by the light, the vertex and the viewer.

It's important to remember that the diffuse attenuation factor of a light is on a per channel basis instead by light. So we have to use one channel for ambient and specular lights and another one for diffuse lights. We put specular lights in the same channel as the ambient light because ambient lights are not attenuated, so we can set the attenuation function to GX\_AF\_SPEC without affecting them. In the second channel we compute the diffuse lights.

The ambient source for both channel is set to GX\_SRC\_REG, and the register values are the global ambient color for the first channel, and zero (which results in no ambient light addition) for the second one.

As for the materials, while OpenGL allows specifying different material colors for the specular and for the ambient lights, in GX we can specify only one material color per each channel. Since we are using the first channel both for the ambient and the specular lights, we need to make a choice between which material color to use: our choice falls onto the ambient color if ambient lighting is enabled, and set it to the specular color only if we have no ambient light. The reason is that, generally, ambient light has a stronger visual impact on the rendered scene, so we prioritize it. This leads to an issue where, if both ambient and specular lights are enabled, the specular light will be computed using the ambient material color, ignoring whatever the client requested for the specular material color; but in most cases this should not result in a noticeable difference.


\subsection {Texture management}

Only 2D and 1D textures are implemented. The formats accepted by the implementation are mainly RGB and RGBA but also other special formats like Luminance and compressed textures. The internal formats accepted by the GPU are shown in Figure~\ref{table:format1}.

\begin{figure}[ht]
\centering

\begin{tabular}{|l|l|l|l|}
\hline
	Format & {Color depth} & {Alpha support} & Compressed \\
\hline
	GX\_TF\_RGB565 & 5/6 bits & No & No\\
\hline
	GX\_TF\_RGBA8  & 8 bits & 8 bits & No\\
\hline
	GX\_TF\_CMPR  & N/A & 1 bit & Yes\\
\hline
	GX\_TF\_IA8  & 8 bits & 8 bits & No \\
\hline
	GX\_TF\_I8  & 8 bits & No & No \\
\hline
	GX\_TF\_A8  & No & 8 bits & No \\
\hline
\end{tabular}
\caption{Texture formats used by OpenGX}
\label{table:format1}
\end {figure}

All formats without transparency are converted to RGB565, which saves memory while having a good color quality. Using RGBA8 would be a waste of memory without any quality gain. For transparent textures we use RGBA8 format, except for luminance-alpha textures, which are converted to IA8 format. Luminance-alpha is a format with one transparency channel and one color channel. All the color components (RGB) have the same value, so it's used in black and white textures with transparency such as text, shadows, etc.

Compressed textures are only used for non-alpha textures because compressed textures only have one bit for transparency (masking), which is not enough. Therefore if the user requests compression on transparent textures the request is ignored. The compression is performed using a DXT compressor by Jonathan Dummer \cite{dxt_cmp} with some specific modifications and big-endian fixing.

Texture levels (used for mipmapping) are implemented in an efficient way. If the user loads one texture level the implementation reserves memory for that level only. If the user loads more levels the memory is resized to accommodate all the levels. GX allows the programmer to specify a level range for a specific texture, so we can have one level textures and multiple level textures.


\subsection {Call lists}

Call lists are mostly implemented in software, storing the command opcodes and parameters into memory and replaying them later. The notable exception is drawing commands, that are compiled into a GX display list and then stored along with the other operations.

The HANDLE\_CALL\_LIST macro is called at the beginning of those GL functions that can be stored into a call list. It takes care of adding the operation to the active call list (if there's one) minimizing the visual impact of call lists on the code base.


\subsection {Stencil test}

The stencil test allows to discard individual fragments depending on the outcome of a comparison between the value stored in the stencil buffer for that fragment and a reference value. The compared values can also be masked with a bitmask before comparison, and the generation of the stencil buffer involves drawing primitives and performing logical and arithmetical operations on the drawn stencil pixels. The stencil test is typically used to render shadows and reflections. Unfortunately, nor the GameCube hardware or the GX APIs provide any support for the stencil test, so we have to emulate it, partially in software, partially with an additional TEV stage.

\subsubsection {Discarding fragments}
\label{sec:stencildiscard}

Let's first see how opengx discards fragments which don't pass the stencil test; for the time being, let's assume that we have managed to build a stencil buffer (in opengx it can be 4 or 8 bits wide, with 4 being the default) and focus only on how to build a TEV stage which does the comparison and discards the fragment. We will be setting up the TEV stage to operate in \emph{compare mode}, where the inputs are combined according to this formula:

    $$ output = d + ((a OP b) ? c : 0 $$

and $OP$ will be either “equal” (\lstinline{GX_TEV_COMP_A8_EQ}) or “greater than” (\lstinline{GX_TEV_COMP_A8_GT}). Our goal is to decide whether a fragment will be drawn or not, so we'll be using the alpha channel as the output, set $d$ to zero and $c$ to the alpha from the previous stage: in this way, depending on the result of the comparison $a OP b$ (we'll see how to set $a$ and $b$ later below), we can control whether display the fragment with its original alpha, or not display it at all.

Note that we'll have to make the Z buffer operate per fragment and not per vertex (by setting \lstinline{GX_SetZCompLoc(GX_DISABLE)}) and set the alpha compare function (\fname{GX\_SetAlphaCompare}) to exclude all fragments having an alpha value of zero: this is important so that the discarded fragments won't update the Z-buffer.

\paragraph{Generating texture coordinates}

The next problem we have to solve is setting up a texture coordinate generation that, once the stencil texture is loaded in our TEV stage, would allow us to read its pixels using screen coordinates; in other words, we want to make it so that for every fragment processed in this stage, its texel would coincide with a screen pixel. This can be achieved by setting up a texture coordinate generation matrix that multiplies the primitive's vector's \emph{position} and transforms that to the exact x and y coordinates that this vertex will occupy on the screen. Such a matrix can be built by concatenating the movel-view matrix with the projection matrix, but we must take into account that such a matrix will transform vertex coordinates into the \lstinline{[-1,1]x[-1,1]} range whereas the TEV expects texture coordinates to be in the \lstinline{[0,1]x[0,1]} range, so we have to concatenate an additional matrix to translate and scale the coordinates by half.

This is relatively simple to do when the projection is an orthographic one, because in that case we set the texture coordinate generator engine to work on 2x4 matrices (since we only deal with affine transformations), so we can use such a matrix to map the \lstinline{[-1,1]x[-1,1]} range to \lstinline{[0,1]x[0,1]}:

\begin{figure}[ht]
\centering
$ \begin{pmatrix}
	0.5 & 0    & 0 & 0.5\\
	0   & -0.5 & 0 & 0.5\\
\end{pmatrix} $
\caption{Matrix to translate the texture coordinates to be in the \lstinline{[0,1]x[0,1]} range: it's composed by two operations: scale by half and translate by half.}
\label{fig:transortho}
\end{figure}

In the case of perspective projection, the $(x, y, z)$ vertex coordinates get divided by the fourth $w$ coordinate before rendering, and $w$ is computed by the fourth row of the perspective matrix (see Figure~\ref{fig:transperspective}), which, being fixed to $(0, 0, -1, 0)$, always renders $w = -z$. The texture coordinate generator in the TEV does not support 4x4 matrices, but only 2x4 and 3x4 ones, so the problem of computing the $w$ coordinate and dividing the $x$, $y$ and $x$ coordinates by it is not trivial.

\begin{figure}[ht]
\centering
\subfigure[The generic OpenGL perspective projection matrix] {
$ \begin{pmatrix}
    a_{1,1} & 0       & a_{1,3} & 0      \\
    0       & a_{2,2} & a_{2,3} & 0      \\
    0       & 0       & a_{3,3} & a_{3,4}\\
    0       & 0       & -1      & 0      \\
\end{pmatrix} $
}
\hspace{1em}
\subfigure[The scale and translation matrix we set on the TEV when a perspective projection is used.] {
$ \begin{pmatrix}
    -0.5 & 0   & 0.5 & 0 \\
    0    & 0.5 & 0.5 & 0 \\
    0    & 0   & 1   & 0 \\
\end{pmatrix} $
}
\caption{Mapping of coordinates when using a perspective projection.}
\label{fig:transperspective}
\end{figure}

Luckily, the texture coordinate generator in the TEV already performs a very similar operation, which we can reuse for our purposes: when fed with a 3x4 matrix, it will generate a vector made of three elements, $(s, t, u)$, of which the last is used to divide the first two in order to obtain the usual $s$ and $t$ texture coordinates (that is, the texture coordinates will effectively be $(s/u, t/u)$). Given that $w = -z$ and that $u$ in our transformation takes the value of $z$, we just need to take care of inverting the sign of our $s$ and $t$ values, which can be easily done by inverting the sign of the $a_{1,1}$ and $a_{2,2}$ elements of our transformation matrix. Unfortunately, there's still another problem: since the division by $u$ is the very final operation that gets performed, we cannot just encode the translation by $0.5$ (which we need in order to remap the vertex coordinates to \lstinline{[0,1]x[0,1]}) as we do in the orthographic case, because this would also get divided by $q$ and produce an incorrect result. The solution to this problem is using a different transformation matrix, which would take into account the fact that the translation factor will also get divided by $q$: if we move the $0.5$ translation elements to the third column, instead of storing them on the fourth one, they will get multiplied by the $z$ coordinate, and this will compensate the division by $q$, being $q = z$:

$$ \begin{pmatrix}
    a_{1,1} & 0   & 0.5 & 0 \\
    0    & a_{2,2} & 0.5 & 0 \\
    0    & 0   & 1   & 0 \\
\end{pmatrix}
\begin{pmatrix}
    x\\
    y\\
    z\\
    1\\
\end{pmatrix}
/ q
=
\begin{pmatrix}
    a_{1,1}x + 0.5z\\
    a_{2,2}y + 0.5z\\
    z\\
    1\\
\end{pmatrix}
/ q
\stackrel{q=z}{=}
\begin{pmatrix}
    \frac{a_{1,1}}{q}x + 0.5\\
    \frac{a_{2,2}}{q}y + 0.5\\
    1\\
    1\\
\end{pmatrix}
$$

This allows us to translate the texture coordinates by half towards the positive direction of the axes.


\paragraph{Comparing stencil texels}

Another issue is how to actually implement the comparison, since the OpenGL specification supports all kinds of arithmetical comparisons, whereas the TEV only supports comparing for equality (\lstinline{GX_TEV_COMP_A8_EQ}) and strict "greater than" (\lstinline{GX_TEV_COMP_A8_GT}); however, since we know that we are operating on integer values, most of this operations can be emulated by inverting the order of the operands in the TEV, or by altering the reference value by ±1, as shown in Figure~\ref{table:stencil1}.

\begin{figure}[ht]
\centering

\begin{tabular}{|l|l|l|}
\hline
    {GL comparison} & {Formula} & {GX comparison} \\
\hline
    GL\_EQUAL & $a = b$ & $a = b$ \\
\hline
	GL\_GREATER & $a > b$ & $a > b$\\
\hline
	GL\_LESS & $a < b$ & $b > a$\\
\hline
	GL\_GEQUAL & $a \geq b$ & $a > b - 1$ \\
\hline
	GL\_LEQUAL & $a \leq b$ & $b + 1 > a$ \\
\hline
\end{tabular}
\caption{Mapping OpenGL stencil comparisons into the TEV. $a$ is the value from the stencil texture, and $b$ is the reference value}
\label{table:stencil1}
\end {figure}

What is missing from that figure is the comparison for not equality (\lstinline{GL_NOTEQUAL}), which simply cannot be implemented using the comparisons provided by the TEV engine. For that reason, if the comparison mode is \lstinline{GL_NOTEQUAL}, opengx builds a special stencil texture in software, whose pixels are set to 1 if the stencil buffer value is different from reference value and 0 otherwise, and then uses the \lstinline{GX_TEV_COMP_A8_GT} (greater than) comparison on this texture's pixels.

Note that while building the stencil texture in software sounds like a non optimal decision (and indeed it is), opengx has to do it in any case, even for those comparisons from Figure~\ref{table:stencil1} supported by the TEV, in order to support the masking step described in the OpenGL specification: that is, OpenGL is not directly comparing the stencil buffer value with the reference value, but both values are \emph{bitwise AND'ed} with a bitmask given by the programmer. This means that before drawing a primitive on which the stencil test must run, opengx will read the stencil buffer and generate a stencil texture that can be used with the comparisons described above. Using the \fname{GX\_ReadBoundingBox} function allows us to minimize the work by updating only the area that changed.

\subsubsection {Drawing the stencil buffer}

When drawing to the stencil buffer, special care must be taken to avoid making this drawing visible, so before performing a stencil draw operation opengx saves the current contents of the EFB to a texture, then loads the previous contents of the stencil buffer into the EFB, perform the stencil rendering, then saves the EFB to the stencil buffer and finally restores the previous contents of the EFB (the ones containing the visible graphics). This is clearly not optimal when the sequence of drawing operations to the screen is intertwined with drawing operations to the stencil buffer (as is often the case), but unfortunately there's no way around this. We could still use the \fname{GX\_ReadBoundingBox} function to keep track of the areas that need saving and restoring, so there might be some room for optimization here.

OpenGL specifies several logical and arithmetical operations that can be applied when rendering to the stencil buffer, with the most used being by far \lstinline{GL_KEEP} and \lstinline{GL_REPLACE}: the former is trivial to implement, since it just means that no drawing must happen on the stencil buffer; the latter is used to replace the stencil buffer's value with the reference value, so this is also not hard to realize by rendering the primitive with no lighting, no texturing, and just fixing the drawing color to the reference value. The \lstinline{GL_ZERO} operation is also easy to implement, being a special case of \lstinline{GL_REPLACE} where instead of using the reference colour we just draw total black. Other operations are not currently implemented (see the \ref{sec:stencillimitations} section for details).

\subsubsection {Limitations}
\label{sec:stencillimitations}

The stencil drawing operations \lstinline{GL_INCR}, \lstinline{GL_DECR}, \lstinline{GL_INCR_WRAP}, \lstinline{GL_DECR_WRAP} and \lstinline{GL_INVERT} are not implemented. It might be possible that at least some of them could be implemented in the hardware by setting an appropriate blending mode. Another posibility is to make these functions not directly render on the stencil buffer, but on yet another offscreen buffer, and then process the results in software. But none of these solutions have been implemented so far, given how rarely these operations are used (at least in public source code indexed by GitHub).


\subsection {Clip planes}

The OpenGL \fname{glClipPlane} function can be used to define a clip plane, which will “cut” the primitives drawn while it is enabled and will show only the part that is on the desired side of the plane. A plane is defined by a linear equation

$$Ax + By + Cz + D = 0$$

and the space of the visible (unclipped) points is defined by the corresponding expression

$$Ax + By + Cz + D \geq 0$$

More than a clipping plane can be defined (OpenGL specifies that implementations should support at least 6 of them), in which case the visible area is the intersection of all these spaces.

The GX API does not offer such a functionality, but this doesn't mean that it cannot be implemented. The first idea was generating a Z-texture for each clipping plane, and then let the depth test exclude the fragments whose Z coordinate did not satisfy the requirements. Special care would have to be taken to support clipping planes intersecting the camera, because in those cases the Z-texture would have to be placed not as the plane itself, but as a semi-plane with the Z set to the maximum value in order to mask out all the unwanted fragments. But the hardest part would have been that of saving the original Z values from the EFB and then restoring them back at the end of the clipping operation, while updating the Z values of those fragments that had actually been drawn. As a solution to this problem was not found, the idea was dropped and the whole feature was deemed to be practically unfeasible.

After implementing the stencil test, though, another implementation idea was born: if we did encode the $A$, $B$, $C$, $D$ coefficient of the clipping plane as the first row of the texture coordinate generation matrix, and used the fragment's position as the input source for the texture coordinate generation, the $s$ texture coordinate we would be generating would be

$$s = Ax + By + Cy + Dw$$

which, given that $w = 1$, would effectively be the computation of the clipping equation: we would get an $s \geq 0$ for those fragments that can be drawn, and $s < 0$ for those who need to be clipped out. Sure, texture coordinates are only meaningful in the range $[0,1]$ so a negative $s$ would wrap or be clamped within this interval, but this can be solved by applying a translation of $0.5$ as the last step of the texture coordinate generation in order to have our conditions changed to $s \geq 0.5$ for the visible fragments and $s < 0.5$ for the invisible ones. At this point, we could use this $s$ coordinate on a very small texture, just two texels wide, with an alpha value of 0 on the left texel and 1 on the right one, and use a TEV stage setup similarly to that described in the \ref{sec:stencildiscard} section to drop all fragments which produce a zero alpha value. Note that we need to set the texture coordinates to be clamped (and not wrapped!), so that any value of $s$ bigger than $0.5$ renders a pixel with full alpha, and any value of $s$ smaller than $0.5$ renders a transparent (discarded) pixel.

Furthermore, we can improve this solution by encoding another clip plane in the second row of the texture coordinate generation matrix (which is responsible for generating the $t$ texture coordinate), and then we can handle two clipping planes in a single TEV stage: we just need to prepare a 2-by-2 pixel texture, having an alpha value of $0$ in three of its pixels, and $1$ in the one pixel for which both $s$ and $t$ are positive.

At the moment opengx hardcodes the maximum number of supported clip planes to 6 (which require 3 TEV stages), but this can be increased if needed. Note that only the required number of TEV stages needed to perform the clipping operations is used.


\subsection {Selection mode}

GL selection mode, also known as "picking mode", is typically used in applications to determine which objects are rendered at a certain screen position, in order to implement mouse interactions. When selection mode is active, drawing primitives do not result in any pixels (or even Z-pixels) being drawn, but instead produce a stack containing the names of the object which would have been drawn.

OpenGX implements selection mode by switching of color and alpha updates, but the Z-buffer is still being drawn to in order to have a way to track the objects being drawn: whenever we want to know if a drawing primitive will result in a hit, we clear the GX bounding box before executing the draw commands and examine it afterwards.

When leaving the selection mode we need to restore the Z buffer to its initial state (which we saved using GX\_CopyTex) by rendering it as a Z-texture.

\subsubsection {Limitations}

The hit information recorded during selection mode should also contain the minimum and maximum values of the Z buffer at the moment that a hit was recorded. This is a relatively expensive operation (there's no shortcut around examining all Z pixels one by one) which OpenGX does not currently perform: not only we haven't found an application using this information, but both the AMD and Mesa drivers on the Linux desktop seem to not deliver it and just set both these values to 0.


\subsection {Programmable pipeline and OpenGL 2.0+}

While the TEV engine can be programmed to achieve very sophisticated graphical effects, there isn't a straightforward way to convert a program written in the GL shading language (GLSL) into a set of TEV instructions. And even if this was possible, executing a GLSL compiler at run-time would be impractical on the GameCube or Wii consoles, due to the amount of RAM and CPU required. The way that OpenGX supports OpenGL 2.0+ programs is by delegating the setup of the TEV to the application developer, in a way though that minimizes the amount of changes to the preexisting application's code.

The idea is to offer a hook system where the application developer can provide a set of function callbacks which will be invoked at different stages of the pipeline, which will take care of setting up the TEV and transformation matrices as needed. Of course, it will not always be possible to get a 1:1 match to what the original GLSL program would output, but given the power of the TEV it should be possible to simulate most effects. To support the case where an application uses several shader programs of the same type, a 32-bit hash\footnote{The hash is currently computed via the \href{https://github.com/aappleby/smhasher/wiki/MurmurHash3}{MurmurHash3} algorithm, which is rather good at avoiding collisions.} of the shader program code is computed when \fname{glLinkProgram} is called and can be used to invoke different setup callbacks depending on the pipeline program being executed.

The OpenGX API exposed to application developers is still subject to change, so we won't be documenting it in detail here (the programs in the \lstinline{examples/} directory can serve as an up-to-date reference), but the general flow is as follows: the hooks can be installed right in the application's \fname{main} function, like this:

\begin{lstlisting}
#if defined(__wii__) || defined(__gamecube__)
#include "opengx_shaders.h"
#endif

int main(int argc, char *argv[])
{
#if defined(__wii__) || defined(__gamecube__)
    setup_opengx_shaders();
#endif
    ...rest of the main() code...
}
\end{lstlisting}

where it's intended that \fname{setup\_opengx\_shaders} if a function defined by the application developer in an extra file \lstinline{opengx_shaders.c} which the build system would link only for the GameCube/Wii targets, and it would be declared in \lstinline{opengx_shaders.h}. Note the both the names of the function and of the files can be changed according to one's own taste, and it's not even needed that the setup happens in a separate file --- it's just a suggestion for minimizing the changes to the application's code. The setup function must then install the callback hooks:

\begin{lstlisting}
static bool compile_shader(GLuint shader)
{
    /* Declare the uniforms and attributes used by this shader program */
    ...
}

static GLenum link_program(GLuint program)
{
    /* Install further callbacks for setting up the transformation
     * matrices and the TEV, depending on the program. */
    ...
}

static const OgxProgramProcessor s_processor = {
    .compile_shader = compile_shader,
    .link_program = link_program,
};

void setup_opengx_shaders()
{
    ogx_shader_register_program_processor(&s_processor);
}
\end{lstlisting}

The callback code should not call any OpenGL functions, as this might interfere with OpenGX, but it can call OpenGL getter methods, typically to inspect the shader programs and to retrieve the location and the value of the uniforms. OpenGL vertex attributes generally don't need any special handling, besides declaring which GX vertex attribute they map to; this is best done in the \fname{compile\_shader} callback, as shown below: OpenGX then passes all enabled vertex attributes to the GX pipe.

\begin{lstlisting}
static bool compile_shader(GLuint shader)
{
    /* We must inform OpenGX of which uniforms are used in this shader
     * program, with their type. */
    ogx_shader_add_uniforms(shader, 4,
        "ModelViewProjectionMatrix", GL_FLOAT_MAT4,
        "NormalMatrix", GL_FLOAT_MAT4,
        "LightSourcePosition", GL_FLOAT_VEC4,
        "MaterialColor", GL_FLOAT_VEC4);

    /* For the attributes, we must also specify the corresponding GX_VA_*
     * attribute that will be mapped to them. */
    ogx_shader_add_attributes(shader, 2,
                              "position", GL_FLOAT_VEC3, GX_VA_POS,
                              "normal", GL_FLOAT_VEC3, GX_VA_NRM);
}
\end{lstlisting}

The main goal of the \fname{link\_program} callback, instead, is that of installing the callbacks that will be invoked every time that the program is executed; the reason why these callbacks are not specified in the same \lstinline{OgxProgramProcessor} structure is that these callbacks are specific to the shader program, so one might install a different set of callbacks for each different pipeline program.

\begin{lstlisting}
static void setup_draw(GLuint program, const OgxDrawData *draw_data,
                       void *user_data)
{
    ShaderData *data = user_data;
    /* Retrieve the uniform values... */
    float colorf[4];
    glGetUniformfv(program, data->mat_color_loc, colorf);
    ...setup the TEV stage here...
}

static void setup_matrices(GLuint program, void *user_data)
{
    ShaderData *data = user_data;
    /* Retrieve and setup the transformation matrices... */
    float m[16];
    glGetUniformfv(program, data->mvp_loc, m);
    ogx_shader_set_mvp_gl(m);
    ...retrieve and setup the normal matrix...
}

static GLenum link_program(GLuint program)
{
    /* Not strictly necessary, but it's a good practice to cache the
     * location of the uniforms that we'll need to use in the callbacks.
     * Note that these uniforms must have been declared by the 
     * compile_shader() callback earlier. */
    ShaderData *data = malloc(sizeof(ShaderData));
    data->mvp_loc =
        glGetUniformLocation(program, "ModelViewProjectionMatrix");
    data->normal_matrix_loc =
        glGetUniformLocation(program, "NormalMatrix");
    data->material_color_loc =
        glGetUniformLocation(program, "MaterialColor");
    data->light_pos_loc =
        glGetUniformLocation(program, "LightSourcePosition");
    ogx_shader_program_set_user_data(program, data, free);

    /* Install the program-specific callbacks */
    ogx_shader_program_set_setup_matrices_cb(program, setup_matrices);
    ogx_shader_program_set_setup_draw_cb(program, setup_draw);
}
\end{lstlisting}

More callbacks types might be added later to support different shader types or to give the application developer more control over the pipeline.

\subsubsection{Writing a fragment shader for OpenGX}

The \fname{setup\_draw} function mentioned in the last snippet is responsible for setting up all the needed TEV stages for drawing the geometry. OpenGX provides a handful of convenience functions to convert some OpenGL types to GX ones (for example, for converting an array of 4 floats into a \lstinline{GXColor}), but the actual instructions for the GPU must be written by the application developer using \lstinline{libogc}'s APIs.

It's important to mention that the developer should not hardcode the stage number or other GPU resource IDs such as texture map IDs and texture coordinate IDs, but should instead allocate the IDs by incrementing the corresponding fields from the \fname{ogx\_gpu\_resources} structure which is made available as a global variable:

\begin{lstlisting}
/* Somewhere in setup_draw(), we "allocate" the IDs: */
uint8_t stage = GX_TEVSTAGE0 + ogx_gpu_resources->tevstage_first++;
uint8_t tex_map = GX_TEXMAP0 + ogx_gpu_resources->texmap_first++;
uint8_t tex_coord = GX_TEXCOORD0 + ogx_gpu_resources->texcoord_first++;

/* Then we use them */
GX_SetTevOrder(stage, tex_coord, tex_map, GX_COLOR0A0);
\end{lstlisting}

By looking at the value of the \lstinline{tevstage_first} field OpenGX will know how many stages have been allocated by the shader, and will call \fname{GX\_SetNumTevStages} with the correct number of parameters (same goes for \fname{GX\_SetNumTexGens}); it's advised that application developers don't invoke these functions, because OpenGX might add additional stages if clipping or stenciling is enabled. Note also that there is no need to deallocate the IDs previously allocate (that is, no decrement operation must be performed on the \lstinline{ogx_gpu_resources} members), because OpenGX will take care of resetting it when needed.

\pagebreak[4]

\begin{thebibliography}{1}

  \bibitem{nintendogx} Nintendo {\em Revolution Graphics Library (GX) Version 1.00}  2006.

  \bibitem{libogc_gx_docs} LibOGC Team? {\em GX Subsystem Reference }

  \bibitem{dx6_mt} Jason Mitchell, Ian Bullard, Michael Tatro  {\em Multitexturing in DirectX 6 }
  
  \bibitem{dxt_cmp} Jonathan Dummer {\em DXT compressor } (http://nothings.org/)
  
\end{thebibliography}

\end{document}
